\documentclass{article}

\usepackage[ngerman]{babel}
\usepackage[utf8]{inputenc}
\usepackage[T1]{fontenc}
\usepackage{hyperref}
\usepackage{csquotes}

\usepackage[
    backend=biber,
    style=apa,
    sortlocale=de_DE,
    natbib=true,
    url=false,
    doi=false,
    sortcites=true,
    sorting=nyt,
    isbn=false,
    hyperref=true,
    backref=false,
    giveninits=false,
    eprint=false]{biblatex}
\addbibresource{../references/bibliography.bib}

\title{Notizen zum Projekt Data Ethics}
\author{Stefani Anastasova}
\date{\today}

\begin{document}
\maketitle

\abstract{
    Dieses Dokument ist eine Sammlung von Notizen zu dem Projekt. Die Struktur innerhalb des
    Projektes ist gleich ausgelegt wie in der Hauptarbeit, somit kann hier einfach geschrieben
    werden, und die Teile die man verwenden möchte, kann man direkt in die Hauptdatei ziehen.
}



\tableofcontents

\section{Einleitung}
Was ist KI?
Künstliche Intelligenz KI bzw. Artificial Intelligence AI ist ein Teilgebiet der Informatik und gilt als "zukunftsweisende Technologie". \\
KI besitzt die Kompetenzen menschliche Fähigkeiten Maschinell nachzuahmen.
Wie funktioniert dies aber? \\
    Training der KI durch Daten\\ 
    1.Trainieren\\
    Im ersten Trainingsschritt wird die Ki mit verschiedenen daten zugeführt, mit diesen Datenmengen wird sie trainiert und kann so beispielsweise lernen Bilder, Gesichter etc. zu erkennen
    so erweitert die Ki ihr können. Sobald der Computer diese Daten hat kann er trainiert werden 
    Es gibt zwei Arten zu lernen:  
    -überwachtes lernen -> hier lernt der Algorithmus von Beispielen aus den Datenmengen wie man ihm gibt, es wird ihm also menschlich geholfen.
    -unüberwachtes lernen-> Hier sucht der Computer selber nach mustern und Lösungen aus den Daten ohne hilfe.

    2. Validierung\\
    Bei diesem Schritt wird überprüft, wie gut der Algorithmus trainiert ist auf Daten denen er nochnie begegnet ist, dies hilft bei der Anpassung der KI. 
    Die Ki lernt also so lange weiter bis keine Verbesserung mehr erreicht werden kann. 
    Das kann helfen um zu helfen, die KI zu verbessern.
    3. Testen
    Im letzen Schritt, wird die Ki schlussendlich getestet und überarbeitet. Damit diese Tests erfolgreich sind muss die KI bestimmte Kriterien erfüllen.
    -Qualität der Daten, das bedeutet die Daten müssen genau und unter anderem auch relevant sein.
    -Ressourcen, hierbei ist die Frage wer den Algorithmus bzw. wenn überhaupt jemand diesen trainiert. 
\\
\\
    \section{Fragestellung}
    Suchmaschinen, Empfehlungsalgorithmen, Sprachaassistenten, heutzutage ist die KI fast überall erhalten und beeinflusst uns sogar unbewusst, denn diese hat sich schnell entwickelt und verbreitet. 
    Sie kann uns durch die Daten helfen, jedoch stellt sich hier auch die Frage wie viel Wer wird auf Datenschutz gelegt? 
    Somit ergibt sich die Fragestellung "Wie kann man sicherstellen, dass KI-Systeme den Datenschutz respektieren?"
    Bei meiner Fragestellung beziehe ich much auf ein spezifischeres KI tool nämlich Chatgpt. 
\\


\section{Beispiel Chatgpt, wie sicher ist die Ki}
    Chatgpt: "ChatGPT hat einen Raketenstart hingelegt. Mittlerweile wird das KI-Tool weltweit genutzt – vom Schüler bis hin zum Firmen-Boss. " zitiert 
    Das KI tool kann bei verschiedenen Fragen und Themen benutzt werden, denn es ist darauf spezialisiert auf menschliche anfragen zu reagieren und darauf Antworten zu geben. 
    Es kann aber auch in vielen weiteren Bereichen verwendet.\\
    Die KI Chatgpt kann nahezu auf alles reagieren und antworten nachdem man sie fragt. 
\\
    Wie sicher sind Gespräche? \\
    Grundsätzlich sind die Interaktionen in Chatgpt sicher, jedoch nur wenn man keine privaten Informationen weitergibt. Ausserdem sind diese Gespräche nicht vertraulich, da die KI diese Daten verwendet.
    Im allgemeinem setzt sich das Unternehmen von Chatgpt OpenAI aber für die Sicherheitsmassnahmen ein.
    Der Chatverlauf wird gespeichert welcher der KI beim Traing behilflich sein kann.
    Welche Daten werden verarbeitet? 
    Für die Nutzung des Chatbots wird eine persönliche Registrierung benötigt, das bedeutet E-Mail-Adresse, Name, Mobilnummer, Geburtsdatum und optional auch Kreditkartendaten als die Bezahlmethode für den Kauf von kostenpflichtigen Diensten.

    Wie viel wert wird auf Datenschutz gelegt
    Der Chatbot speichert die personenbezogenen Daten um diesen zu verbessern und weiterhin zu trainieren, wobei die Organisation OpenAI zugibt gewisse Daten mit Partnern zu teilen, wenn dies dem Chatbot zur Verbesserung hilft.
    Obwohl die KI die Daten für einen bestimmten Zweck erfasst, wird empfohlen keine persönlichen Daten im chat preiszugeben.

Datenschutzherausforderungen bei Chatgpt 
Unabsichtliche Offenlegung der persönlichen Informationen
Während einer interaktion mit dem Chatbot, können Nutzer unbeabsichtigt sensible, persönliche Infos teilen,  

Wie kann man also sicherstellen, dass dieses KI system den persönlichen Datenschutz respektiert?
Durch die Einfuehrung von Massnahmen für die Sicherheit des Datenschutz.
Es gibt also Datenschutzbestimmungen welche darauf ausgerichtet sind die Privatsphäre der Nutzer zu schützen.
1. Einwilligung: Bevor eine Einwilligung eingegangen wird, müssen sich die Nutzer bewusst sein, das die KI ihre daten verarbeitet.
Die Einwilligung sollte freiwillig, klar und spezifisch sein.
2.Transparenz: Es ist wichtig, dass KI-Systeme klar kommunizieren, welche Daten sie erheben, wie sie diese nutzen und wie lange sie sie lagern. Benutzer sollen Zugriff auf Datenverarbeitungsinformationen haben.
3.Zweckbindung: Die erhobenen Daten dürfen nur zu dem angegebenen Zweck genutzt werden, für den sie erhoben wurden, und nicht zu anderen Zwecken, sofern keine erneute Zustimmung vorliegt.
4.Datensicherheit: Um die gespeicherten Daten vor unbefugtem Zugriff, Verlust oder Diebstahl zu schützen, sind KI-Systeme mit geeigneten Sicherheitsvorkehrungen verpflichtet.
Die Rechte der Nutzer umfassen den Zugang zu ihren gespeicherten Daten, die Korrektur, das Löschen oder die Einschränkung ihrer Verarbeitung.



\printbibliography

\end{document}
