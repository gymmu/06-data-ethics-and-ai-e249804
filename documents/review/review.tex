\documentclass{article}

\usepackage[ngerman]{babel}
\usepackage[utf8]{inputenc}
\usepackage[T1]{fontenc}
\usepackage{hyperref}
\usepackage{csquotes}

\usepackage[
    backend=biber,
    style=apa,
    sortlocale=de_DE,
    natbib=true,
    url=false,
    doi=false,
    sortcites=true,
    sorting=nyt,
    isbn=false,
    hyperref=true,
    backref=false,
    giveninits=false,
    eprint=false]{biblatex}
\addbibresource{../references/bibliography.bib}

\title{Review des Papers "Ethik im Umgang mit Daten" von Levin Kaya}
\author{Stefani Anastasova}
\date{\today}

\begin{document}
\maketitle

\abstract{
    In diesem Review wird folgende Arbeit zum Thema Daten und Ethik von Levin Kaya analysiert und kritisch beurteilt mit positiven sowohl auch negativen Aspekten.

}
\section{Positive Aspekte}

Diese Arbeit hat eine gute Struktur und somit eine gute Themenreihenfolge. Die Gliederung ist somit gut und verständlich aufgebaut.  Die Abschnitte in der Einleitung sind informativ und nicht allzu kompliziert erläutert worden, 
was dem Leser keine Verständnisprobleme bereiten sollte. Die KI wird somit präzise erklärt und gibt einen Einblick in die Welt der Technologie in Bezug auf die künstliche Intelligenz.
Dein Satzbau und Wortwahl sind beide gut gewählt und du verwendest eine klare und deutliche Sprache somit sind auch etwas komplexere Textblöcke gut zu verstehen. Ich finde ausserdem deine Fragestellung spannend und sehr aktuell, dadurch ist die Arbeit von hohem Interesse.
Zudem ist deine Datei ästhetisch schön gestaltet und beinhaltet auch ein Titelbild. Man merkt, dass du dich kritisch mit dem Thema auseinandergesetzt hast und versucht hast dies in deiner Arbeit zu zeigen, 
denn man kann erkennen das du dich mit diesem Thema interessiert umgegangen bist und meiner Meinung nach erfolgreich einen super Aufsatz zu deiner Fragestellung gemeistert hast. 


\section{Negative Aspekte}
Ich werde in diesem Abschnitt die Aspekte bennenen, die mir etwas unvorteilhaft aufgefallen sind.
Es gibt einige kleine Grammatikfehler welche sich in den Texten eingeschlichen haben wie zum Beispiel "speziefschen", ausserdem habe ich das Gefühl, dass sich bestimmte Themen wiederholen oder ähnlich klingen, dass neigt dazu,
dass das Interesse des Lesers sich minimiert. Ausserdem würde ich noch ein wenig tiefer in das Thema der ethischen Diskussion gehen und diese mit guten Beispielen stärken. 


\section{Verbesserungsvorschläge}
Ich würde dir empfehlen, die Texte nochmals gründlich durchzulesen und nach Fehlern zu überprüfen, damit der Text attraktiver zum durchlesen ist. 
Zudem sollte der Text auf wiederholungen geprüft werden, denn dies würde den Text kompakter und interessanter machen. 
Die Arbeit könnte zusätzlich relevanter und zeitgemäßer werden, wenn Sie mehr über die neuesten Entwicklungen und aktuellen Forschungsergebnisse in den Bereichen Künstliche Intelligenz und Ethik schreiben könnten.

\section{Fazit}
Im grossen und ganzen finde ich diese Arbeit ist dir gut gelungen, man sieht, dass du dich mit diesem Thema intensiv befasst und 
die Fragestellung gut erläutert hast.it einigen sprachlichen und inhaltlichen Verbesserungen könnte es jedoch noch überzeugender und professioneller wirken jedoch ist es meiner Meinung nach
eine informative und interessante Arbeit geworden, die das Thema Daten, Ethik und KI umfasst.   

\printbibliography

\end{document}
